\documentclass[conference]{IEEEtran}
\IEEEoverridecommandlockouts
% The preceding line is only needed to identify funding in the first footnote. If that is unneeded, please comment it out.
\usepackage{cite}
\usepackage{amsmath,amssymb,amsfonts,amsthm}
\usepackage{algorithmic}
\usepackage{graphicx}
\usepackage{textcomp}
\usepackage{xcolor}
\usepackage{tikz}
\ifCLASSOPTIONcompsoc
\usepackage[caption=false,font=normalsize,labelfont=sf,textfont=sf]{subfig}
\else
\usepackage[caption=false,font=footnotesize]{subfig}
\fi

\newtheorem{theorem}{Theorem}[section]

\theoremstyle{definition}
\newtheorem{definition}{Definition}[section]

\theoremstyle{remark}
\newtheorem{exmp}{Example}
 
\newtheorem{corollary}{Corollary}

\begin{document}
    \title{Notes on Hamiltonian Mechanics}

    \author{\IEEEauthorblockN{Zisheng Ye}}

    \maketitle
    
    Hamiltonian mechanics is geometry in phase space. Phase space has the structure of a symplectic manifold.The group of symplectic diffeomorphisms acts on phase space. The concepts and theorems of hamiltonian mehcanics are invariant under this group.
    
\end{document}