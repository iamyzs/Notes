\documentclass[letterpaper,12pt]{article}
\usepackage[top=1in,left=1in,right=1in,bottom=1in]{geometry}
\usepackage[utf8]{inputenc}
\usepackage{amsmath,amssymb,amsfonts,amsthm}

\allowdisplaybreaks

\numberwithin{equation}{section}

\title{Notes on SPH}
\author{Zisheng Ye}
\date{April 2019}

\begin{document}

\maketitle

\section{SPH Concept and Essential Formulation}
The continuous SPH integral representation for $f(\mathbf{x})$ can be written in the following form of discretized particle approximation
\begin{align}
    f(\mathbf{x}) =& \int_{\Omega} f(\mathbf{x}^\prime) W(\mathbf{x} - \mathbf{x}^\prime, h) \mathrm{d} \mathbf{x}^\prime \nonumber \\
    \approx& \sum_{j = 1}^N f(\mathbf{x}_j) W(\mathbf{x} - \mathbf{x}_j, h) \Delta V_j \nonumber \\
    =& \sum_{j = 1}^N f(\mathbf{x}_j) W(\mathbf{x} - \mathbf{x}_j, h) \dfrac1{\rho_j}(\rho_j \Delta V_j) \nonumber \\
    =& \sum_{j = 1}^N f(\mathbf{x}_j) W(\mathbf{x} - \mathbf{x}_j, h) \dfrac1{\rho_j} m_j \nonumber \\
    =& \sum_{j = 1}^N \dfrac{m_j}{\rho_j} f(\mathbf{x}_j) W(\mathbf{x} - \mathbf{x}_j, h) \label{eq:discretized_particle_approximation}
\end{align}

For a given particle $i$, according to the particle approximation, the value of a function and its derivative for particle $i$ are approximated as
\begin{equation}
    \langle f(\mathbf{x}_i) \rangle = \sum_{j = 1}^N \dfrac{m_j}{\rho_j} f(\mathbf{\mathbf{x}_j}) W_{ij}
\end{equation}
\begin{equation}
    \langle \nabla \cdot f(\mathbf{x}_i) \rangle = \sum_{j = 1}^N \dfrac{m_j}{\rho_j} f(\mathbf{x}_j) \cdot \nabla_i W_{ij}
\end{equation}
\begin{equation}
    W_{ij} = W(\mathbf{x}_i - \mathbf{x}_j, h) = W(| \mathbf{x}_i - \mathbf{x}_j |, h)
\end{equation}
\begin{equation}
    \nabla_i W_{ij} = \dfrac{\mathbf{x}_i - \mathbf{x}_j}{r_{ij}} \dfrac{\partial W_{ij}}{\partial r_{ij}} = \dfrac{x_{ij}}{r_{ij}} \dfrac{\partial W_{ij}}{\partial r_{ij}}
\end{equation}

Summationa density approach: or the SPH approximation for the density is obtained
\begin{equation}
    \rho_i = \sum_{j = 1}^N m_j W_{ij}
\end{equation}

Here are some techniques in deriving SPH formulations
\begin{align}
    \nabla \cdot f(\mathbf{x}_i) =& \dfrac{1}{\rho_i} [ \nabla \cdot (\rho_i f(\mathbf{x}_i) ) - f(\mathbf{x}_i) \cdot \nabla \rho_i ] \nonumber \\
    =& \dfrac{1}{\rho_i} \left[ \sum_{j=1}^N \dfrac{m_j}{\rho_j} \rho_j f(\mathbf{x}_j) \cdot \nabla_i W_{ij} - f(\mathbf{x}_i) \sum_{j=1}^N m_j \cdot \nabla_i W_{ij} \right] \nonumber \\
    =& \dfrac1{\rho_i} \left[ \sum_{j = 1}^N m_j \left[ f(\mathbf{x}_j) - f(\mathbf{x}_i) \right] \cdot \nabla_i W_{ij} \right]
\end{align}
\begin{align}
    \nabla \cdot f(\mathbf{x}_i) =& \rho_i \left[\nabla \cdot \left( \dfrac{f(\mathbf{x}_i)}{\rho_i} \right) + \dfrac{f(\mathbf{x}_i)}{\rho_i^2} \cdot \nabla \rho_i \right] \nonumber \\
    =& \rho_i \left[ \sum_{j=1}^N \dfrac{m_j}{\rho_j} \dfrac{f(\mathbf{x}_j)}{\rho_j} \cdot \nabla_i W_{ij} + \dfrac{f(\mathbf{x}_i)}{\rho_i^2} \sum_{j=1}^N m_j \cdot \nabla_i W_{ij} \right] \nonumber \\
    =& \rho_i \left[ \sum_{j=1}^N m_j \left[ \dfrac{f(\mathbf{x}_j)}{\rho_j^2} + \dfrac{f(\mathbf{x}_i)}{\rho_i^2} \right] \cdot \nabla_i W_{ij} \right]
\end{align}

There are some properties of SPH formulations as
\begin{equation}
    \langle f_1 + f_2 \rangle = \langle f_1 \rangle + \langle f_2 \rangle = \langle f_2 + f_1 \rangle
\end{equation}
\begin{equation}
    \langle f_1 f_2 \rangle = \langle f_1 \rangle \langle f_2 \rangle = \langle f_2 f_1 \rangle
\end{equation}

By definition, the support domain for a field point at $\mathbf{x} = (x, y, z)$ is the domain where the information for all the points inside this domain is used to determine the information at the points at $\mathbf{x}$. The influence domain is defined as a domain where a node exerts its influences. Hence the influence domain is associated with a node in the meshfree methods, and the support domain goes with any field point $\mathbf{x}$, which can be, but does not necessarily have to be a node.

\section{Constructing Smoothing Functions}
In the original SPH paper, the following bell-shaped function was used as the smoothing function
\begin{equation}
    W(\mathbf{x} - \mathbf{x}^\prime, h) = W(R, h) = \alpha_d
    \left\{
        \begin{aligned}
            & (1 + 3R) (1 - R)^3 & \quad R \leq 1 \\
            & 0 & \quad R > 1
        \end{aligned}
    \right.
\end{equation}
where $\alpha_d$ is $5/4 h$, $5/\pi h^2$ and $105/16\pi h^3$ in one-, two- and three-dimensional space, respectively. $R$ is the relative distance between two points $R = \dfrac{r}{h} = \dfrac{|\mathbf{x} - \mathbf{x}^\prime|}{h}$.

The Guassian kernel
\begin{align}
    W(R, h) = \alpha_d e^{-R^2}
\end{align}
where $\alpha_d$ is $1/\pi^{1/2} h$, $1/\pi h^2$ and $1/\pi^{3/2} h^3$, respectively, in one-, two- and three-dimensional space. The Gaussian kernal is sufficiently smooth even for high orders of derivatives, and is regarded as a golden slection since it is very stable and accurate especially for disordered particles. Note that it is computationally more expensive since it can take a longer distance for the kernel to approach zero, especially for higher order derivatives of the smoothing function. This can result in a large support domain with an inclusion of more particles for the particle approximation, and therefore leads to a larger bandwidth in the discrete system matrix.

From the cubic spline functions known as the B-spline function, we have
\begin{equation}
    W(R, h) = \alpha_d
    \left\{
        \begin{aligned}
            & \dfrac23 - R^2 + \dfrac12 R^3 & \quad 0 \leq R < 1 \\
            & \dfrac16(2 - R)^3 & \quad 1 \leq R < 2 \\
            & 0 & \quad R \geq 2
        \end{aligned}
    \right.
\end{equation}
In one-, two- and three-dimensional space, $\alpha_d = 1/h$, $15/7\pi h^2$ and $3/2\pi h^3$, respectively.

The quartic spline
\begin{equation}
    W(R, h) = \alpha_d
    \left\{
        \begin{aligned}
            & (R+2.5)^4 - 5(R+1.5)^4 + 10(R+0.5)^4 & \quad 0 \leq R < 0.5 \\
            & (2.5-R)^4 - 5(1.5-R)^4 & \quad 0.5 \leq R < 1.5 \\
            & (2.5-R)^4 & \quad 1.5 \leq R < 2 \\
            & 0 & \quad R \geq 2
        \end{aligned}
    \right.
\end{equation}
where $\alpha_d$ is $1/24h$ in one-dimensional space.

The quinitic spline
\begin{equation}
    W(R, h) = \alpha_d
    \left\{
        \begin{aligned}
            & (3-R)^5 - 6(2-R)^5 + 15(1-R)^5 & \quad 0 \leq R < 1 \\
            & (3-R)^5 - 6(2-R)^5 & \quad 1 \leq R < 2 \\
            & (3-R)^5 & \quad 2 \leq R < 3 \\
            & 0 & \quad R \geq 3
        \end{aligned}
    \right.
\end{equation}
where $\alpha_d$ is $120/h$, $7/478\pi h^2$ and $3/359\pi h^3$ in one-, two- and three-dimensional space, respectively.

Another quadratic smoothing function to simulate the high velocity impact problem is
\begin{equation}
    W(R, h) = \alpha_d (\dfrac{3}{16} R^2 - \dfrac34 R + \dfrac34) \quad 0 \leq R \leq 2
\end{equation}
where in one-, two- and three0dimensional space, $\alpha_d = 1/h$, $2/\pi h^2$ and $5/4 \pi h^3$, respectively. Unlike other smoothing functions, the derivative of this quadratic smoothing function always increases as the particles move closer, and always decreases as they move apart. This is an important improvement over the cubic spline function and is reported to relieve the problem of compressive instability.

Some higher order smoothing functions that are devised from lower order forms, such as the superGaussian kernel
\begin{equation}
    W(R, h) = \alpha_d (\dfrac32 - R^2) e^{-R^2} \quad 0 \leq R \leq 2
\end{equation}
where $\alpha_d = 1/\sqrt{\pi}$ in one-dimensional space. One disadvantage of the high order smoothing function is that the kernel is negative in some region of its support domain which may lead to unphysical results for hydrodynamic problems.

Apply the Taylor series expansion of $f(\mathbf{x}^\prime)$ in the vicinty of $\mathbf{x}$ yields
\begin{align}
    f(\mathbf{x}^\prime) =& f(\mathbf{x}) + f^\prime(\mathbf{x}) (\mathbf{x}^\prime  - \mathbf{x}) + \dfrac12 f^{\prime\prime}(\mathbf{X}) (\mathbf{x}^\prime  - \mathbf{x})^2 + \dots \nonumber \\
    =& \sum_{k=0}^n \dfrac{(-1)^k h^k f^{(k)}(\mathbf{x})}{k!} \left( \dfrac{\mathbf{x} - \mathbf{x}^\prime}{h} \right)^k + r_n\left( \dfrac{\mathbf{x} - \mathbf{x}^\prime}{h} \right) \label{eq:taylor_expansion}
\end{align}

Substituting Eq.\ref{eq:taylor_expansion} into Eq.\ref{eq:discretized_particle_approximation} leads to
\begin{align}
    f(\mathbf{x}) =& \int_{\Omega} \sum_{k=0}^n \dfrac{(-1)^k h^k f^{(k)}(\mathbf{x})}{k!} \left( \dfrac{\mathbf{x} - \mathbf{x}^\prime}{h} \right)^k W(\mathbf{x} - \mathbf{x}^\prime, h) \mathrm{d} \mathbf{x}^\prime + r_n\left( \dfrac{\mathbf{x} - \mathbf{x}^\prime}{h} \right) \nonumber \\
    =& \sum_{k=0}^n \dfrac{(-1)^k h^k f^{(k)}(\mathbf{x})}{k!} \int_{\Omega} \left( \dfrac{\mathbf{x} - \mathbf{x}^\prime}{h} \right)^k W(\mathbf{x} - \mathbf{x}^\prime, h) \mathrm{d} \mathbf{x}^\prime + r_n\left( \dfrac{\mathbf{x} - \mathbf{x}^\prime}{h} \right) \nonumber \\
    =& \sum_{k=0}^n A_k f^{(k)}(\mathbf{x}) + r_n\left( \dfrac{\mathbf{x} - \mathbf{x}^\prime}{h} \right)
\end{align}
where
\begin{align}
    A_k = \dfrac{(-1)^k h^k }{k!} \int_{\Omega} \left( \dfrac{\mathbf{x} - \mathbf{x}^\prime}{h} \right)^k W(\mathbf{x} - \mathbf{x}^\prime, h) \mathrm{d} \mathbf{x}^\prime
\end{align}
Since the LHS and RHS of the previous equation should be equal, the following conditions for the smoothing function $W$ can be obtained
\begin{equation}
    \left.
    \begin{aligned}
        A_0 =& \int_{\Omega} W(\mathbf{x} - \mathbf{x}^\prime, h) \mathrm{d} \mathbf{x}^\prime = 1 \\
        A_1 =& -h \int_{\Omega} \left( \dfrac{\mathbf{x} - \mathbf{x}^\prime}{h} \right) W(\mathbf{x} - \mathbf{x}^\prime, h) \mathrm{d} \mathbf{x}^\prime = 0 \\
        A_2 =& \dfrac{h^2}{2!} \int_{\Omega} \left( \dfrac{\mathbf{x} - \mathbf{x}^\prime}{h} \right)^2 W(\mathbf{x} - \mathbf{x}^\prime, h) \mathrm{d} \mathbf{x}^\prime = 0 \\
        \vdots & \\
        A_n =& \dfrac{(-1)^n h^n}{n!} \int_{\Omega} \left( \dfrac{\mathbf{x} - \mathbf{x}^\prime}{h} \right)^n W(\mathbf{x} - \mathbf{x}^\prime, h) \mathrm{d} \mathbf{x}^\prime = 0
    \end{aligned}
    \right\}
\end{equation}
These conditions can be further written in the simplified expressions in terms of the $k-$th moments $M_k$ of the smoothing function.
\begin{equation}
    \left.
    \begin{aligned}
        M_0 =& \int_{\Omega} W(\mathbf{x} - \mathbf{x}^\prime, h) \mathrm{d} \mathbf{x}^\prime = 1 \\
        M_1 =& \int_{\Omega} \left( \dfrac{\mathbf{x} - \mathbf{x}^\prime}{h} \right) W(\mathbf{x} - \mathbf{x}^\prime, h) \mathrm{d} \mathbf{x}^\prime = 0 \\
        M_2 =& \int_{\Omega} \left( \dfrac{\mathbf{x} - \mathbf{x}^\prime}{h} \right)^2 W(\mathbf{x} - \mathbf{x}^\prime, h) \mathrm{d} \mathbf{x}^\prime = 0 \\
        \vdots & \\
        M_n =& \int_{\Omega} \left( \dfrac{\mathbf{x} - \mathbf{x}^\prime}{h} \right)^n W(\mathbf{x} - \mathbf{x}^\prime, h) \mathrm{d} \mathbf{x}^\prime = 0
    \end{aligned}
    \right\}
\end{equation}

\end{document}
