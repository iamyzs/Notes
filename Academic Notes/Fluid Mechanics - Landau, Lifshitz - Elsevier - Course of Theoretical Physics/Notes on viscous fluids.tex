\documentclass[conference]{IEEEtran}
\IEEEoverridecommandlockouts
% The preceding line is only needed to identify funding in the first footnote. If that is unneeded, please comment it out.
\usepackage{cite}
\usepackage{amsmath,amssymb,amsfonts,amsthm}
\usepackage{algorithmic}
\usepackage{graphicx}
\usepackage{textcomp}
\usepackage{xcolor}
\usepackage{tikz}
\ifCLASSOPTIONcompsoc
\usepackage[caption=false,font=normalsize,labelfont=sf,textfont=sf]{subfig}
\else
\usepackage[caption=false,font=footnotesize]{subfig}
\fi

\newtheorem{theorem}{Theorem}[section]

\theoremstyle{definition}
\newtheorem{definition}{Definition}[section]

\theoremstyle{remark}
\newtheorem{exmp}{Example}
 
\newtheorem{corollary}{Corollary}

\renewcommand{\theequation}{\thesection.\arabic{equation}}

\usepackage{chngcntr}

\counterwithin*{equation}{section}

\allowdisplaybreaks

\begin{document}
    \title{Notes on viscous fluids}

    \author{\IEEEauthorblockN{Zisheng Ye}}

    \maketitle

    \section{THe equations of motion of a viscous fluid}

    The effect of energy dissipation occurs during the motion of a fluid, on that motion itself. This process is the result of the thermodynamic irreversibility of the motion. This irreversibility always occurs to some extent, and is due to internal friction (viscosity) and thermal conduction.

    In order to obtain the equations describing the motion of a viscous fluid, some additional terms in the equation of motion of an ideal fluid have to be included. The equation of continuity, as seen from its derivation, is equally valid for any fluid, whether viscous or not. Euler's equation, on the other hand, requires modification.

    The Euler's equation can be written in the form
    \[
        \dfrac{\partial}{\partial t} (\rho v_i) = -\dfrac{\partial \Pi_{ik}}{\partial x_k}
    \]
    where $\Pi_{ik}$ is the momentum flux density tensor. Previously, the momentum flux represents a completely reversible transfer of momentum, due simply to the mechanical transport of the different particles of fluid from place to place and to the pressure forces acting in the fluid. The viscosity (internal friction) causes another, irreversible, transfer of momentum from points where the velocity is large to those where it is small.

    The equation of motion of a viscous fluid may therefore be obtained by adding a term $\sigma^\prime_{ik}$ which gives the irreversible \emph{viscous} transfer of momentum in the fluid. Thus we write the momentum flux density tensor in a viscous fluid in the form
    \begin{equation}
        \Pi_{ik} = p\delta_{ik} + \rho v_i v_k - \sigma^\prime_{ik} = -\sigma_{ik} + \rho v_i v_j
    \end{equation}
    The tensor
    \begin{equation}
        \sigma_{ik} = -p\delta_{ik} + \sigma^\prime_{ik}
    \end{equation}
    is called the \emph{stress tensor}, and $\sigma^\prime_{ik}$ the \emph{viscous stress tensor}. $\sigma_{ik}$ gives the part of the momentum flux that is not due to the direct transfer of momentum with the mass of moving fluid.

    Processes of internal friction occur in a fluid only when different fluid particles move with different velocities, so that there is a relative motion between various parts of the fluid. Hence $\sigma^\prime_{ik}$ must depend on the space derivatives of the velocity. If the velocity gradients are small, we may suppose that the momentum transfer due to viscosity depends only on the first derivatives of the velocity. To the same approximation, $\sigma^\prime_{ik}$ may be supposed a linear function of the derivatives $\partial v_i / \partial x_k$. There can be no terms in $\sigma^\prime_{ik}$ independent of $\partial v_i / \partial x_k$, since $\sigma^\prime_{ik}$ must vanish for $\mathbf{v} = \text{constant}$. Next, we notice that $\sigma^\prime_{ik}$ must also vanish when the whole fluid is in uniform rotation, since it is clear that in such a motion no internal friction occurs in the fluid. In uniform rotation with angular velocity $\boldsymbol{\omega}$, the velocity $\mathbf{v}$ is equal to the vector product $\boldsymbol{\omega} \times \mathbf{r}$. The sums
    \[
        \dfrac{\partial v_i}{\partial x_k} + \dfrac{\partial v_k}{\partial x_i}
    \]
    are linear combinations of the derivatives $\partial {v_i} / \partial x_k$, and vanish when $\mathbf{v} = \boldsymbol{\omega} \times \mathbf{r}$. Hence $\sigma^\prime_{ik}$ must contain just these symmetrical combinations of the derivatives $\partial v_i / \partial x_k$.

    The most general tensor of rank two satisfying the above conditions is 
    \begin{equation}
        \sigma^\prime_{ik} = \eta \left( \dfrac{\partial v_i}{\partial x_k} + \dfrac{\partial v_k}{\partial x_i} - \dfrac23 \delta_{ik} \dfrac{\partial v_l}{\partial x_l} \right) + \zeta \delta_{ik} \dfrac{\partial v_l}{\partial x_l}
    \end{equation}
    with coefficients $\eta$ and $\zeta$ independent of the velocity. In making this statement, the fluid is isotropic, as a result of which its properties must be described by scalar quantities only. The terms here are arranged so that the expression in parentheses has the property of vanishing on contraction with respect to $i$ and $k$. The constants $\eta$ and $\zeta$ are called \emph{coefficients of viscosity}, and $\zeta$ often the \emph{second viscosity}. These two coefficients are both postive.

    The equations of motion of a viscous fluid can now be obtained by simply adding the expressions $\partial \sigma^\prime_{ik} / \partial x_k$ to the right-hand side of Euler's equation
    \begin{align}
        & \rho \left( \dfrac{\partial v_i}{\partial t} + v_k \dfrac{\partial v_i}{\partial x_k} \right) = -\dfrac{\partial p}{\partial x_i} + \dfrac{\partial}{\partial x_i} \left( \zeta \dfrac{\partial v_l}{\partial x_l} \right) \nonumber \\ 
        & \quad + \dfrac{\partial}{\partial x_k} \left\{ \eta \left( \dfrac{\partial v_i}{\partial x_k} + \dfrac{\partial v_k}{\partial x_i} -\dfrac23 \delta_{ik} \dfrac{\partial v_l}{\partial x_l} \right) \right\}
        \label{eq:navier_stokes_index}
    \end{align}
    This is the most general form of the equations of motion of a viscous fluid. The quantities $\eta$ and $\zeta$ are functions of pressure and temperature. In general, $p$ and $T$, and therefore the fluid, so that $\eta$ and $\zeta$ cannot be taken outside the gradient operator.

    In most cases, however, the viscosity coefficients do not change noticeably in the fluid, and they may be regarded as constant. We then have equation Eq.~\ref{eq:navier_stokes_index}, in vector form, as
    \begin{align}
        \rho \left[ \dfrac{\partial \mathbf{v}}{\partial t} + (\mathbf{v} \cdot \nabla) \mathbf{v} \right] =& -\nabla p + \eta \Delta \mathbf{v} \nonumber \\
        & + (\zeta + \dfrac13 \eta) \nabla (\nabla \cdot \mathbf{v})
        \label{eq:navier_stokes_vector}
    \end{align}
    This is called the \emph{Navier-Stokes equation}. It becomes considerably simpler if the fluid may be regarded as incompressible, so that $\nabla \cdot \mathbf{v} = 0$, and the last term on the right of Eq.~\ref{eq:navier_stokes_vector} is zero. In discussing viscous fluids, we shall always regard them as incompressible, and accordingly use the equation of motion in the form
    \begin{equation}
        \dfrac{\partial \mathbf{v}}{\partial t} + (\mathbf{v} \cdot \nabla) \mathbf{v} = -\dfrac1\rho \nabla p + \dfrac{\eta}{\rho} \Delta \mathbf{v}
        \label{eq:ns_impressible}
    \end{equation}
    The stress tensor in an incompressible fluid takes the simple form
    \begin{equation}
        \sigma_{ik} = - p \delta_{ik} + \eta \left( \dfrac{\partial v_i}{\partial x_k} + \dfrac{\partial v_k}{\partial x_i} \right)
    \end{equation}

    The viscosity of an incompressible fluid is determined by only one coefficient. Since most fluids may be regarded as pratically incompressible, it is this coefficient $\eta$ which is generally of importance. The ratio
    \[
        \nu = \dfrac{\eta}{\rho}
    \]
    is called the \emph{kinematic viscosity} (while $\eta$ itself is called the \emph{dynamic viscosity}). The pressure can be eliminated in Eq.~\ref{eq:ns_impressible} in the same way as from Euler's equation and by taking the curl of both sides, 
    \[
        \dfrac{\partial}{\partial t} (\nabla \times \mathbf{v}) = \nabla \times (\mathbf{v} \times (\nabla \times \mathbf{v})) + \nu \Delta (\nabla \times \mathbf{v})
    \]
    Since the fluid is incompressible, the equation can be transformed by expanding product in the first term on the right and using the equation $\nabla \cdot \mathbf{v} = 0$
    \begin{align}
        \dfrac{\partial}{\partial t} (\nabla \times \mathbf{v}) =& \mathbf{v} \nabla \cdot (\nabla \times \mathbf{v}) - \mathbf{v} \cdot \nabla (\nabla \times \mathbf{v}) \nonumber \\
        &- (\nabla \times \mathbf{v}) \nabla \cdot \mathbf{v} + (\nabla \times \mathbf{v}) \cdot \nabla \mathbf{v} \nonumber \\
        &+ \nu \Delta (\nabla \times \mathbf{v}) \nonumber \\
        \dfrac{\partial}{\partial t} (\nabla \times \mathbf{v}) &+ \mathbf{v} \cdot \nabla (\nabla \times \mathbf{v}) -  (\nabla \times \mathbf{v}) \cdot \nabla \mathbf{v} \nonumber \\
        =& \nu \Delta (\nabla \times \mathbf{v})
    \end{align}

    When the velocity distribution is known, the pressure distribution in the fluid can be recovered by solving the Poisson-type equation:
    \begin{equation}
        \Delta p = -\rho \dfrac{\partial v_i}{\partial x_k} \dfrac{\partial v_k}{\partial x_i} = -\rho \dfrac{\partial^2 v_i v_k}{\partial x_i \partial x_k}
    \end{equation}

    Then consider the boundary condition on the equations of motion of a viscous fluid. There are always forces of attraction between a viscous fluid and the surface of solid body, and these forces have the result that the layer of fluid immediately adjacent to the surface is brought completely to rest and \emph{adheres} tothe surface. Accordingly, the boundary conditions on the equations of motion of a viscous fluid require that the fluid velocity should vanish at fixed solid surfaces
    \begin{equation}
        \mathbf{v} = 0
    \end{equation}

    It is emphasized that both the normal and tangential velocity component must vanish.

    In the general case of a moving surface, the velocity $\mathbf{v}$ must be equal to the velocity of the surface.

    It is easy to write down an expression for the force acting on a solid surface bounding the fluid. The force acting on an element of the surface is just the momentum flux through this element. The momentum flux through the surface element $\mathrm{d} \mathbf{f}$ is
    \[
        \Pi_{ik} \mathrm{d} f_k = (\rho v_i v_k - \sigma_{ik}) \mathrm{d} f_k
    \]

    In determining the force acting on the surface, each surface element must be considered in a frame of reference in which it is at rest. The force is equal to the momentum flux only when the surface is fixed. Writing $\mathrm{d} f_k$ in the form $\mathrm{d} f_k = n_k \mathrm{d} f$, where $\mathbf{n}$ is a unit vector along the normal, and recalling that $\mathbf{v} = 0$ at a solid surface, we find that the force $\mathbf{P}$ acting on unit surface area is
    \begin{equation}
        P_i = -\sigma_{ik} n_k = p n_i - \sigma^\prime_{ik} n_k
    \end{equation}
    The first term is the ordinary pressure of the fluid, while the second is the force of friction, due to the viscosity, acting on the surface. It must be emphasized that $\mathbf{n}$ above is a unit vector along the outward normal to the fluid, along the inward normal to the solid surface.

    If we have a surface of separation between two immiscible fluids, the conditions at the surface are that the velocities of the fluids must be equal and the forces which they exert on each other must be equal and opposite. The latter condition is written
    \[
        n_{1, k} \sigma_{1, ik} + n_{2, k} \sigma_{2, ik} = 0
    \]
    where the suffixes 1, 2 refer to the two fluids. The normal vecttor $\mathbf{n}_1$ and $\mathbf{n}_2$ are in opposite directions, $\mathbf{n}_1 = -\mathbf{n}_2$, so that it can be written
    \begin{equation}
        n_k \sigma_{1, ik} = n_k \sigma{2, ik}
    \end{equation}
    At a free surface of the fluid the condition
    \begin{equation}
        \sigma_{ik} n_{k}  \equiv \sigma_{ik}^\prime n_k - p n_i = 0
    \end{equation}
    must hold.

    In cylindrical polar coordinates $r, \phi, z$ the components of the stress tensor are
    \begin{equation}
        \begin{aligned}
        \sigma_{rr} =& -p + 2 \eta \dfrac{\partial v_r}{\partial r} \\
        \sigma_{\phi\phi} =& -p + 2\eta \left( \dfrac{1}{r} \dfrac{\partial v_{\phi}}{\partial \phi} + \dfrac{v_r}{r} \right) \\
        \sigma_{zz} =& -p + 2\eta \dfrac{\partial v_z}{\partial z} \\
        \sigma_{r\phi} =& \eta \left( \dfrac1r \dfrac{\partial v_r}{\partial \phi} + \dfrac{\partial v_\phi}{\partial r} - \dfrac{v_\phi}{r} \right) \\
        \sigma_{\phi z} =& \eta \left( \dfrac{\partial v_\phi}{z} + \dfrac1r \dfrac{\partial v_z}{\partial \phi} \right) \\
        \sigma_{zr} =& \eta \left( \dfrac{\partial v_z}{\partial r} + \dfrac{\partial v_r}{\partial z} \right)
        \end{aligned}
    \end{equation}
    The three components of the Navier-Stokes equation are
    \begin{align}
        \dfrac{\partial v_r}{\partial t} + (\mathbf{v} \cdot \nabla) v_r - \dfrac{v_{\phi}^2}{r} =& -\dfrac1\rho \dfrac{\partial p}{\partial r} + \nu \left( \Delta v_r - \dfrac2{r^2} \dfrac{\partial v_{\phi}}{\partial \phi} - \dfrac{v_r}{r^2} \right) \nonumber \\
        \dfrac{\partial v_{\phi}}{\partial t} + (\mathbf{v} \cdot \nabla) v_{\phi} + \dfrac{v_{\phi}v_r}{r} =& -\dfrac1{\rho r} \dfrac{\partial p}{\partial \phi} + \nu \left( \Delta v_{\phi} - \dfrac2{r^2} \dfrac{\partial v_r}{\partial \phi} - \dfrac{v_{\phi}}{r^2} \right) \nonumber \\
        \dfrac{\partial v_z}{\partial t} + (\mathbf{v} \cdot \nabla) v_z =& -\dfrac1\rho \dfrac{\partial p}{\partial z} + \nu \Delta v_z
    \end{align}
    where
    \begin{align*}
        (\mathbf{v} \cdot \nabla) f =& v_r \dfrac{\partial f}{\partial r} + \dfrac{v_{\phi}}{r} \dfrac{\partial f}{\partial \phi} + v_z \dfrac{\partial f}{\partial z} \\
        \Delta f =& \dfrac1r \dfrac{\partial}{\partial r}\left( r \dfrac{\partial f}{\partial r} \right) + \dfrac1{r^2} \dfrac{\partial^2 f}{\partial \phi^2} + \dfrac{\partial^2 f}{\partial z^2}
    \end{align*}
    The equation of continuity is
    \begin{equation}
        \dfrac1r \dfrac{\partial (rv_r)}{\partial r} + \dfrac1r \dfrac{\partial v_\phi}{\partial \phi} + \dfrac{\partial v_z}{\partial z} = 0
    \end{equation}

    In spherical polar coordinates $r, \phi, \theta$, the stress tensor is
    \begin{equation}
        \begin{aligned}
            \sigma_{rr} =& -p + 2\eta \dfrac{\partial v_r}{\partial r} \\
            \sigma_{\phi\phi} =& -p + 2\eta \left( \dfrac1{r\sin \theta} \dfrac{\partial v_\phi}{\partial \phi} + \dfrac{v_r}{r} + \dfrac{v_\theta \cot \theta}{r} \right) \\
            \sigma_{\theta\theta} =& -p + 2\eta \left( \dfrac1r \dfrac{\partial v_\theta}{\partial \theta} + \dfrac{v_r}{r} \right) \\
            \sigma_{r\theta} =& \eta \left( \dfrac1r \dfrac{\partial v_r}{\partial \theta} + \dfrac{\partial v_\theta}{\partial r} - \dfrac{v_\theta}{r} \right) \\
            \sigma_{\theta\phi} =& \eta \left( \dfrac1{r\sin \theta} \dfrac{\partial v_\theta}{\partial \phi} + \dfrac1r \dfrac{\partial v_\phi}{\partial \theta} - \dfrac{v_\phi \cot \theta}{r} \right) \\
            \sigma_{\phi r} =& \eta \left( \dfrac{\partial v_\phi}{\partial r} + \dfrac1{r \sin \theta}\dfrac{\partial v_r}{\partial \phi} - \dfrac{v_\phi}{r} \right)
        \end{aligned}
    \end{equation}
    while the Navier-Stokes equations are
    \begin{align}
        \dfrac{\partial v_r}{\partial t} +& (\mathbf{v} \cdot \nabla) v_r - \dfrac{v_\theta^2 + v_\phi^2}r = -\dfrac1\rho \dfrac{\partial p}{\partial r} \nonumber \\
        +& \nu \left[ \Delta v_r - \dfrac2{r^2\sin^2\theta} \dfrac{\partial (v_\theta \sin\theta)}{\partial \theta} - \dfrac2{r^2\sin\theta} \dfrac{\partial v_\phi}{\partial \phi} - \dfrac{2v_r}{r^2} \right] \nonumber \\
        \dfrac{\partial v_\theta}{\partial t} +& (\mathbf{v} \cdot \nabla) v_\theta + \dfrac{v_r v_\theta}r - \dfrac{v_\phi^2 \cot \theta}r = -\dfrac1{\rho r} \dfrac{\partial p}{\partial \theta} \\
        +& \nu \left[ \Delta v_\theta - \dfrac{2\cos\theta}{r^2\sin^\theta} \dfrac{\partial v_\phi}{\partial \phi} + \dfrac2{r^2} \dfrac{\partial v_r}{\partial \theta} - \dfrac{v_\theta}{r^2 \sin^2 \theta} \right] \nonumber \\
        \dfrac{\partial v_\phi}{\partial t} +& (\mathbf{v} \cdot \nabla) v_\phi + \dfrac{v_r v_\phi}{r} + \dfrac{v_\theta v_\phi \cot \theta}r = -\dfrac1{\rho r \sin \theta} \dfrac{\partial p}{\partial \phi} \nonumber \\
        +& \nu \left[ \Delta v_\phi + \dfrac2{r^2\sin\theta} \dfrac{\partial v_r}{\partial \phi} + \dfrac{2\cos\theta}{r^2\sin^2\theta} \dfrac{\partial v_\theta}{\partial \phi} - \dfrac{v_\phi}{r^2 \sin^2\theta} \right] \nonumber
    \end{align}
    where
    \begin{align*}
        &(\mathbf{v} \cdot \nabla) f = v_r \dfrac{\partial f}{\partial r} + \dfrac{v_\theta}r \dfrac{\partial f}{\partial \theta} + \dfrac{v_\phi}{r \sin\theta} \dfrac{\partial f}{\partial \phi} \\
        &\Delta f = \dfrac1{r^2} \dfrac{\partial}{\partial r} \left( r^2 \dfrac{\partial f}{\partial r} \right) + \dfrac1{r^2 \sin \theta} \left( \sin\theta \dfrac{\partial f}{\partial \theta} +\dfrac1{r^2\sin^2\theta} \dfrac{\partial^2 f}{\partial \phi^2} \right)
    \end{align*}
    The equation of continuity is
    \begin{equation}
        \dfrac1{r^2} \dfrac{\partial (r^2 v_r}{\partial r} + \dfrac1{r\sin\theta} \dfrac{\partial (v_\theta \sin\theta)}{\partial \theta} + \dfrac1{r\sin\theta} \dfrac{\partial v_\phi}{\partial \phi} = 0
    \end{equation}

    \section{Energy dissipation in an incompressible fluid}
    The presence of viscosity results in the dissipation of energy, which is finally transformed into heat. The calculation of the energy dissipation is especially simple for an incompressible fluid.

    THe total kinetic energy of an incompressible fluid is
    \[
        E_{\text{kin}} = \dfrac12 \rho \int v^2 \mathrm{d} V
    \]
    Take the time derivative of this energy and substituting for $\dfrac{\partial v_i}{\partial t}$ the expression for it given by the Navier-Stokes equation:
    \begin{equation*}
        \dfrac{\partial v_i}{\partial t} = -v_k \dfrac{\partial v_i}{\partial x_k} - \dfrac1\rho \dfrac{\partial p}{\partial x_i} + \dfrac1\rho \dfrac{\partial \sigma^\prime_{ik}}{\partial x_k}
    \end{equation*}
    The result is
    \begin{align*}
        & \dfrac{\partial}{\partial t} \left( \dfrac12 \rho v^2 \right) \\
        =& \rho \mathbf{v} \dfrac{\partial \mathbf{v}}{\partial t} \\
        =& -\rho \mathbf{v} \cdot (\mathbf{v} \cdot \nabla \mathbf{v}) - \mathbf{v} \cdot \nabla p + v_i \dfrac{\partial \sigma^\prime_{ik}}{\partial x_k} \\
        =& -\rho \mathbf{v} \cdot \nabla \left(\dfrac12 v^2 + \dfrac{p}{\rho} \right) + \nabla \cdot (\mathbf{v} \cdot \boldsymbol{\sigma}^\prime) - \sigma_{ik}^\prime \dfrac{\partial v_i}{\partial x_k}
    \end{align*}
    Here $\mathbf{v} \cdot \boldsymbol{\sigma}^\prime$ whose components are $v_i \sigma^\prime_{ik}$. Since $\nabla \cdot \mathbf{v} = 0$ for an incompresisble fluid, we can write the first term on the right as a divergence
    \begin{align*}
        & \dfrac{\partial}{\partial t} (\dfrac12 \rho v^2) \\
        =& -\nabla \cdot \left[ \rho \mathbf{v} \left( \dfrac12v^2 + \dfrac{p}{\rho} \right) - \mathbf{v} \cdot \boldsymbol{\sigma}^\prime \right] - \sigma_{ik}^\prime \dfrac{\partial v_i}{\partial x_k}
    \end{align*}
    The expression in brackets is just the energy flux density in the fluid: the term $\rho \mathbf{v} (\dfrac12 v^2 + \dfrac{p}{\rho})$ is the energy flux due to the actual transfer of fluid mass, and is the same as the energy flux in an ideal fluid. The second term, $\mathbf{v} \cdot \boldsymbol{\sigma}^\prime$, is the energy flux due to processes of internal friction. For the presence of viscosity results in a momentum flux $\sigma_{ik}^\prime$; a transfer of momentum, however, always involves a transfer of energy, and the energy flux is clearly equal to the scalar product of the momentum flux and veloctiy.

    If integrate it over some volume $V$,
    \begin{equation}
        \begin{aligned}
            &\dfrac{\partial}{\partial t} \int \dfrac12 \rho v^2 \mathrm{d} V \\
            =&-\oint \left[ \rho \mathbf{v} \left( \dfrac12 v^2 + \dfrac{p}{\rho} \right)  - \mathbf{v} \cdot \boldsymbol{\sigma}^\prime \right] \cdot \mathrm{d} \mathbf{f} - \int \sigma_{ik}^\prime \dfrac{\partial v_i}{\partial x_k} \mathrm{d} V
        \end{aligned}
    \end{equation}
    The first term on the right gives the rate of change of kinetic energy of the fluid in $V$ owing to the energy flux through the surface bounding $V$. The integral in the second term is consequently the decrease per unit time in the kinetic energy owing to dissipation.

    Consider the motion of the fluid in a system of coordinates such that the fluid is at rest at infinity. For a fluid enclosed in a finite volume, the surface integral again vanishes, because the velocities at the surface vanishes. Therefore, if the integral is extended to the whole volume of the fluid, the surface integral vanishes, and the energy dissipated per unit time in the whole fluid is bo be
    \begin{equation}
        \dot{E_{\text{kin}}} = - \int \sigma_{ik}^\prime \dfrac{\partial v_i}{\partial x_k} \mathrm{d} V = \dfrac12 \int \sigma_{ik}^\prime \left( \dfrac{\partial v_i}{\partial x_k} + \dfrac{\partial v_k}{\partial x_i} \right) \mathrm{d} V
    \end{equation}
    since the tensor $\sigma_{ik}^\prime$ is symmetrical. In incompresisble fluids, the energy dissipation in an incompressible gluid is
    \begin{equation}
        \dot{E}_{\mathrm{kin}} = \dfrac12 \eta \int \left( \dfrac{\partial v_i}{\partial x_k} + \dfrac{\partial v_k}{\partial x_i} \right)^2 \mathrm{d} V
    \end{equation}
    
    The dissipation leads to a decrease in the mechanical energy. Since $\dot{E}_{\mathrm{kin}} < 0$ and the integral is always positive. Therefore, the viscosity coefficient $\eta$ is always positive.

    \section{Flow with small Reynolds numbers}
    The Navier-Stokes equation is considerably simplified in the case of flow with small Reynolds numbers. For steady flow of am incompressible fluid, the equation is
    \begin{equation*}
        (\mathbf{v} \cdot \nabla) \mathbf{v} = -\dfrac1\rho \nabla p + \dfrac{\eta}{\rho} \Delta \mathbf{v}
    \end{equation*}
    The ratio of the term $(\mathbf{v} \cdot \nabla) \mathbf{v}$ and term $\dfrac{\eta}{\rho} \Delta \mathbf{v}$ is just the Reynolds number. Hence the term $(\mathbf{v} \cdot \nabla) \mathbf{v}$ may be neglected if the Reynold number is small, and the equation of motion reduces to a linear equation
    \begin{equation}
        \eta \Delta \mathbf{v} - \nabla p = 0
        \label{eq:equation_of_motion_stokes}
    \end{equation}
    Together with the equation of continuity
    \begin{equation}
        \nabla \cdot \mathbf{v} = 0
    \end{equation}
    it completely determines the motion. By taking the curl of the equation of motion,
    \begin{equation}
        \Delta (\nabla \times \mathbf{v}) = 0
        \label{eq:curl_of_equation_of_stokes_flow}
    \end{equation}

    As an example, consider rectilinear and uniform motion of a sphere in a viscous fluid. The problem of the motion of a sphere is exactly equivalent to that of flow past a fixed sphere, the fluid having a given velocity $\mathbf{v}_{\mathrm{inf}}$ at infinity. The velocity distribution in the first problem is obtained from that in the second problem by simply subtracting the velocity $\mathbf{v}_{\mathrm{inf}}$; the fluid is then at rest at infinity, while the sphere moves with velocity $-\mathbf{v}_{\mathrm{inf}}$. If we regard the flow as steady, we must, speak of the flow past a fixed sphere. When the sphere moves, the velocity of the fluid at any point in space varies with time.

    Since $\nabla \times (\mathbf{v} - \mathbf{v}_{\mathrm{inf}}) = \nabla \times \mathbf{v} = 0$, $\mathbf{v} - \mathbf{v}_{\mathrm{inf}}$ can be expressed as the curl of some vector $\mathbf{A}$:
    \begin{equation*}
        \mathbf{v} - \mathbf{v}_{\mathrm{inf}} = \nabla \times \mathbf{A}
    \end{equation*}
    with $\nabla \times \mathbf{A}$ equal to zero at infinity. The vector $\mathbf{A}$ must be axial, in order for its cuil to be polar, like the velocity. In flow past a sphere, a completely symmetrical body, there is no preferred direction other than that of $\mathbf{v}_{\mathrm{inf}}$. This parameter $\mathbf{v}_{\mathrm{inf}}$ must appear linearly in $\mathbf{A}$, because the equation of motion and its boundary conditions are linear. The general form of a vector function $\mathbf{A}(\mathbf{r})$ satisfying all these requirements is $\mathbf{A} = f^\prime (r) \mathbf{n} \times \mathbf{v}_{\mathrm{inf}}$, where $\mathbf{n}$ is a unit vector parallel to the position vector $\mathbf{r}$ (the origin being taken at the centre of the sphere), and $f^\prime(r)$ is a scalar function of r. The product $f^\prime(r) \mathbf{n}$ can be represented as the gradient of another function $f(r)$. Thus the velocity could be expressed in the form:
    \begin{equation}
        \mathbf{v} = \mathbf{v}_{\mathrm{inf}} + \nabla \times (\nabla f \times \mathbf{v}_\mathrm{inf}) = \mathbf{v}_{\mathrm{inf}} + \nabla \times \nabla (f \mathbf{v}_{\mathrm{inf}})
        \label{eq:velocity_in_f}
    \end{equation}

    Next to determine the function $f$. Since
    \begin{align*}
        \nabla \times \mathbf{v} =& \nabla \times \nabla \times \nabla \times (f \mathbf{v}_{\mathrm{inf}}) \\
        =& \nabla\left( \nabla \cdot \left( \nabla \times (f\mathbf{v}_\mathrm{inf}) \right) \right) - \Delta (\nabla \times (f \mathbf{v}_\mathrm{inf})) \\
        =& -\Delta (\nabla \times (f\mathbf{v}_\mathrm{inf}))
    \end{align*}

    Use Eq. \ref{eq:curl_of_equation_of_stokes_flow} takes the form
    \begin{align*}
        &\Delta^2 (\nabla \times (f\mathbf{v}_\mathrm{inf})) \\
        =& \Delta^2 (\nabla f \times \mathbf{v}_\mathrm{inf}) \\
        =& (\Delta^2 \nabla f) \times \mathbf{v}_\mathrm{inf} \\
        =& 0
    \end{align*}

    It follows from this that
    \begin{equation}
        \Delta^2 \nabla f = 0
    \end{equation}
    A first integration gives
    \begin{equation}
        \Delta f^2 = \mathrm{const.}
    \end{equation}
    Since the velocity difference $\mathbf{v} - \mathbf{v}_\mathrm{inf}$ must vanish at infinity, and so must its derivatives, the constant must be zero. The expression $\Delta^2 f$ contains fourth derivatives of $f$, whilst the velocity is given in terms of the second derivatives of $f$. Thus
    \begin{equation*}
        \Delta^2 f \equiv \dfrac1{r^2} \dfrac{\mathrm{d}}{\mathrm{d} r} \left( r^2 \dfrac{\mathrm{d}}{\mathrm{d} r} \right) \Delta f = 0
    \end{equation*}
    Hence
    \[
        \Delta f = 2\dfrac{a}{r} + c
    \]
    The constant $c$ must be zero if the velocity $\mathbf{v} - \mathbf{v}_\mathrm{inf}$ is to vanish at infinity. Therefore
    \begin{equation}
        f = ar + \dfrac{b}{r}
    \end{equation}
    The additive constant is omitted, since it is immaterial (the velocity being given by derivatives of $f$).

    Finally, the expression of velocity is
    \begin{equation}
        \mathbf{v} = \mathbf{v}_\mathrm{inf} - a \dfrac{\mathbf{v}_\mathrm{inf} + \mathbf{n} (\mathbf{v}_\mathrm{inf} \cdot \mathbf{n})} r + b \dfrac{3\mathbf{n} (\mathbf{v}_\mathrm{inf} \cdot \mathbf{n} - \mathbf{v}_\mathrm{inf})}{r^3}
    \end{equation}

    The constants $a$ and $b$ have to be determined from the boundary conditions: at the surface of the sphere $(r = R)$, $\mathbf{v} = 0$,
    \begin{equation*}
        \mathbf{v}_\mathrm{inf} \left( \dfrac{a}{R} + \dfrac{b}{R^3} - 1 \right) + \mathbf{n} (\mathbf{v}_\mathrm{inf} \cdot \mathbf{n}) \left( -\dfrac{a}{R} + \dfrac{3b}{R^3} \right) = 0
    \end{equation*}
    Since this equation must hold for all $\mathbf{n}$, the coefficients of $\mathbf{v}_\mathrm{inf}$ and $\mathbf{n} (\mathbf{v}_\mathrm{inf} \cdot \mathbf{n})$ must each vanish. Hence $a = \dfrac34R$, $b = \dfrac14R^3$. Thus finally
    \begin{align}
        f =& \dfrac34 Rr + \dfrac{R^3}{4r} \\
        \mathbf{v} =& -\dfrac34 R \dfrac{\mathbf{v}_\mathrm{inf} + \mathbf{n} (\mathbf{v}_\mathrm{inf} \cdot \mathbf{n})}r - \dfrac14 R^3 \dfrac{\mathrm{v} - 3\mathbf{n} (\mathbf{v}_\mathrm{inf} \cdot \mathbf{n})}{r^3} + \mathbf{v}_\mathrm{inf}
    \end{align}
    or, in spherical polar components with the axis parallel to $\mathbf{v}_\mathrm{inf}$,
    \begin{equation}
        \left\{
            \begin{aligned}
                v_r =& v_\mathrm{inf} \cos \theta \left[ 1 - \dfrac{3R}{2r} + \dfrac{R^3}{2r^3} \right] \\
                v_\theta =& -v_\mathrm{inf} \sin \theta \left[ 1 - \dfrac{3R}{4r} - \dfrac{R^3}{4r^3} \right]
            \end{aligned}
        \right.
        \label{eq:velocity_in_spherical_coordinates}
    \end{equation}
    This gives the velocity distribution about the moving shpere. To determine the pressure, substitute Eq. \ref{eq:velocity_in_f} in Eq. \ref{eq:equation_of_motion_stokes}:
    \begin{align*}
        \nabla p =& \eta \Delta \mathbf{v} \nonumber \\
        =& \eta \Delta \nabla \times (\nabla \times (f\mathbf{v}_\mathrm{inf})) \nonumber \\
        =& \eta \Delta (\nabla (\nabla \cdot(f \mathbf{v}_\mathrm{inf})) - \mathbf{v}_\mathrm{inf} \Delta f)
    \end{align*}
    But $\Delta^2 f = 0$, and so
    \begin{equation*}
        \nabla p = \nabla [\eta \Delta \nabla \cdot (f \mathbf{v}_\mathrm{inf})] = \nabla (\eta \mathbf{v}_\mathrm{inf} \cdot \nabla (\Delta f))
    \end{equation*}
    Hence
    \begin{equation}
        p = \eta \mathbf{v}_\mathrm{inf} \cdot \nabla  (\Delta f) + p_0
    \end{equation}
    where $p_0$ is the fluid pressure at infinity. Substitution for $f$ leads to the final expression
    \begin{equation}
        p = p_0 - \dfrac32 \eta \dfrac{\mathbf{v}_\mathrm{inf} \cdot \mathbf{n}}{r^2} R
    \end{equation}

    Using the above formulae, the force $\mathbf{F}$ exerted on the sphere by the moving fluid (or, what is the same thing, the drag on the sphere as it moves through the fluid) can be calculated. Take spherical polar coordinates with axis parallel to $\mathbf{v}_\mathrm{inf}$; by symmetry, all quantities are functions only of $r$ and of the polar angle $\theta$. The force $\mathbf{F}$ is evidently parallel to the velocity $\mathbf{v}_\mathrm{inf}$. Substituting Eq. \ref{eq:velocity_in_spherical_coordinates} in the following formulae
    \begin{equation*}
        \sigma_{rr}^\prime = 2\eta \dfrac{\partial v_r}{\partial r}, \quad \sigma_{r\theta}^\prime = \eta \left( \dfrac1r \dfrac{\partial v_r}{\partial \theta} + \dfrac{\partial v_\theta}{\partial r} - \dfrac{v_\theta}{r} \right)
    \end{equation*}
    , at the surface of the sphere
    \begin{equation*}
        \sigma_{rr}^\prime = 0, \quad \sigma_{r\theta}^\prime = -\dfrac{3\eta}{2R} v_\mathrm{inf} \sin \theta
    \end{equation*}
    while the pressure $p$ is $p_0 - \dfrac{3\eta}{2R} v_\mathrm{inf} \cos \theta$. The components, normal and tangential to the surface, of the force on an element of the surface of the sphere, and projecting these components on the direction of $\mathbf{v}_\mathrm{inf}$, we find
    \begin{equation}
        F = \oint(-p \cos \theta + \sigma_{rr}^\prime \cos \theta - \sigma_{r\theta}^\prime \sin \theta) \mathrm{d} f
        \label{eq:integral_of_F}
    \end{equation}
    Hence, the integral \ref{eq:integral_of_F} reduces to $F = \dfrac{3\eta v_\mathrm{inf}}{2R} \oint \mathrm{d} f$. In this way we finally arrive at \emph{Stokes' formula} for the drag on a sphere moving slowing in a fluid:
    \begin{equation}
        F = 6\pi \eta R v_\mathrm{inf}
        \label{eq:Stokes_formula}
    \end{equation}
    
    The drag is proportional to the velocity and linear size of the body. This could have been forseen from dimensional arguments: the fluid density $\rho$ does not appear in the approximate, Stoke's equations, and so the force $F$ which they give must be expressed only in terms of $\eta$, $v_\mathrm{inf}$ and $R$; from these, only one combination with the with the dimension of force can be formed, namely the product $\eta R v_\mathrm{inf}$.

    A similar dependence occurs for slowly moving bodies with other shapes. The direction of the drag on a body of arbitrary shape is not the same as that of the velocity; the general form of the dependence of $\mathbf{F}$ on $\mathbf{v}_\mathrm{inf}$ can be written as 
    \begin{equation}
        F_i = \eta a_{ik} u_k
        \label{eq:dependece_F_on_u}
    \end{equation}
    where $a_{ik}$ is a tensor of rank two, independent of the velocity. It is important to note that this tensor is symmetrical, a result which holds in the linear approximation with respect to velocity, and is a particular case of a general law valid for slow motion accompanied by dissipative processes.

    The above solution of the problem of flow past a sphere is not valid at large distances, even if the Reynolds number is small. To see this, let us estimate the term in $\mathbf{v} \cdot \nabla) \mathbf{v}$ neglected in Eq. \ref{eq:equation_of_motion_stokes}. At large distances, $\mathbf{v} \approx \mathbf{v}_\mathrm{inf}$; the velocity derivatives there are of the order of $\dfrac{v_\mathrm{inf} R}{r^2}$. Hence $(\mathbf{v} \cdot \nabla) \mathbf{v} \sim \dfrac{v_\mathrm{inf}^2 R}{r^2}$. The terms retained in Eq. \ref{eq:equation_of_motion_stokes} are of the order of $\eta\dfrac{R v_\mathrm{inf}}{\rho r^3} \gg \dfrac{v_\mathrm{inf}^2 R}{r^2}$ is satisfied only for distances such that
    \begin{equation}
        r \ll \dfrac{v}{v_\mathrm{inf}}
    \end{equation}
    At greater distances, the terms neglected are not negligile, and the velocity distribution so found is incorrect.

    To find the velocity distribution at large distances from the body, the omitted term has to be recovered. Since at these distances $\mathbf{v}$ is almost the same as $\mathbf{v}_\mathrm{inf}$. Therefore, approximately replace $\mathbf{v} \cdot \nabla$ by $\mathbf{v}_\mathrm{inf} \cdot \nabla$. We then find for the velocity at large distances the linear equation
    \begin{equation}
        (\mathbf{v}_\mathrm{inf} \cdot \nabla) \mathbf{v} = -\dfrac1\rho \nabla p + \nu \Delta \mathbf{v}
        \label{eq:Oseen_equation}
    \end{equation}
    (C. W. Oseen 1910). The velocity distribution thus obtained can be used to derive a more accurate formula for the drag on the sphere, which includes the next term in the expansion of the drag in powers of the Reynolds number $\mathrm{R} = \dfrac{v_\mathrm{inf}R}{v}$
    \begin{equation}
        F = 6\pi \eta v_\mathrm{inf} R (1 + \dfrac{3v_\mathrm{inf}}{8 v})
        \label{eq:drag_per_unit_length_on_cylinder}
    \end{equation}
    In solving the problem of flow past an infinite cylinder at right angles to its axis, Oseen's equation has to be used from the start; Eq. \ref{eq:equation_of_motion_stokes} has in this case no solution satisfying the boundary conditions at the surface of the cylinder and also at infinity. The drag per unit length of the cylinder is found to be
    \begin{equation}
        F = \dfrac{4n\eta v_\mathrm{inf}}{\frac12 - C - \log(\dfrac{v_\mathrm{inf}}{4v})} = \dfrac{4\pi \eta v_\mathrm{inf}}{\log(3 \cdot \frac{70v}{v_\mathrm{inf}R})}
    \end{equation}
    where $C$ is Euler's constant (H. Lamb 1911).

    Another comment should be made regarding the problem of flow past a sphere. The replacement of $\mathbf{v}$ by $\mathbf{v}_\mathrm{inf}$ in the non-linear term in \ref{eq:Oseen_equation} is valid at large distances from the sphere, $r \gg R$. It is therefore natural that Oseen's equation, while correctly refining the picture of flow at large distances, does not do the same at short distances. This is evident from the fact that the solution of \ref{eq:Oseen_equation} which satisfies the necessary conditions at infinity does not satisfy the exact condition that the velocity be zero on the surface of the sphere, which is met only by the zero-order term in the expansion of the velocity in powers of the Reynolds number and not even by the first-order term.

    It might therefore seem at first sight that the solution of Oseen's equation cannot be used for a valid calculation of the correction term in the drag. This is not so, however, for the following reason. The contribution to $\mathbf{F}$ from the motion of the fluid at short distances (for which $v_\mathrm{inf} \ll v / r$) has to be expandable in powers of $\mathbf{v}_\mathrm{inf}$. The first non-zero correction term in the vector $\mathbf{F}$ arising from this contribution therefore has to be proportional to $\mathbf{v}_\mathbf{inf} v_\mathbf{inf}^2$, and gives a second-order correction relative to the Reynolds number; it thus does not affect the first correction in Eq. \ref{eq:drag_per_unit_length_on_cylinder}.

    Further corrections to Stokes' formula and a valid refinement of the flow pattern at short distances can  not be obtained by a direct solution of Eq. \ref{eq:Oseen_equation}. It is considerable methodological interest in deriving and analysing a consistent perturbation theory for solving problemsof vicous flow at small Reynolds numbers.

    To show explicity the small parameter $\mathrm{R}$, the Reynolds number, use the dimensionless velocity and position vector $\mathbf{v}^\prime = \dfrac{\mathbf{v}}{v_\mathrm{inf}}$, $\mathbf{r}^\prime = \dfrac{\mathbf{r}}{v_\mathrm{inf}}$, and in the rest of this section, denote them by $\mathbf{v}$ and $\mathbf{r}$ without the primes. The exact solution of the equation of motion is then
    \begin{equation}
        \mathrm{R} \nabla \times (\mathbf{v} \times (\nabla \times \mathbf{v})) + \Delta (\nabla \times \mathbf{v}) = 0
    \end{equation}
    Distinguish two regions of space around the sphere: the near region with $r \ll 1/R$, and the far region with $r \gg 1$. THese together cover all space, overlapping in the intermediate range
    \begin{equation}
        1/\mathrm{R} \gg r \gg 1
    \end{equation}

    In a consistent perturbation theory, the initial approximation in the near region is the Stokes qpproximation, by neglecting the term which contains the factor $\mathrm{R}$. In dimensionless variables, it is
    \begin{align}
        v_r^{(1)} =& \cos \theta \left(1 - \dfrac3{2r} + \dfrac1{2r^3} \right) \nonumber \\
        v_\theta^{(1)} =& -\sin \theta \left( 1 - \dfrac3{4r} -\dfrac1{4r^3} \right) \nonumber \\
        r \ll& \dfrac{1}{\mathrm{R}}
        \label{eq:r_ll_1/R}
    \end{align}
    the superscript $(1)$ denoting the first approximation.

    The first approximation in the far region is simply the constant $\mathbf{v}^{(1)} = \boldsymbol{\nu}$ corresponding to the unperturbed uniform incoming flow. Substitution of $\mathbf{v} = \boldsymbol{\nu} + \mathbf{v}^{(2)}$ gives for $\mathbf{v}^{(2)}$ Oseen's equation
    \begin{equation}
        \mathrm{R} \nabla \times (\mathbf{v} \times (\nabla \times \mathbf{v}^{(2)})) + \Delta (\nabla \times \mathbf{v}^{(2)}) = 0
    \end{equation}
    The solution must satisfy the condition that the velocity $\mathbf{v}^{(2)}$ be zero at infinity and the condition for joining to the solution above in the intermediate range. The latter excludes, in particular, solutions that increase too rapidly with decreasing $r$. The appropriate solution is
    \begin{align}
        v_r^{(1)} + v_r^{(2)} =& \cos \theta +\dfrac3{2r^2 \mathrm{R}} \nonumber \\
        &- \dfrac3{2r^2 \mathrm{R}}\left[1 + \dfrac12 r \mathrm{R}(1 + \cos \theta) \right]e^{-\frac12 r \mathrm{R} (1 - \cos \theta)} \nonumber \\
        v_\theta^{(1)} + v_\theta^{(2)} =& -\sin \theta + \dfrac3{4r} \sin 
        \theta e^{-\frac12r\mathrm{R}(1 - \cos\theta)} \nonumber \\
        r \gg& 1
    \end{align}
    Note that the varibale for the far region is really the product $\rho = r \mathrm{R}$, not the radial coordinate $r$ itself. When this variable is used, $\mathrm{R}$ disappears, in accordance with the fact that when $r \gtrsim 1/R$ the vicous and inertia terms in the equation become comparable in order of magnitude. The number $\mathrm{R}$ occurs in the solution only through the boundary condition for joining to that in the near region. the expansion of $\mathbf{v}(\mathbf{r})$ in the far region is therefore an expansion in term of $\rho$, contain $\mathrm{R}$ as a factor. In the intermediate range $r\mathrm{R} \ll 1$ and the above expression can be expanded in powers of this variabla. As far as the first two terms (apart from the uniform flow),
    \begin{equation}
        \left\{
            \begin{aligned}
                v_r =& \cos \theta \left( 1 - \dfrac{3}{2r} \right) + \dfrac{3\mathrm{R}}{16}(1 - \cos \theta)(1 + 3\cos \theta) \\
                v_\theta =& -\sin \theta \left( 1 - \dfrac3{4r} \right) - \dfrac{3\mathrm{R}}8 \sin\theta (1 - \cos \theta)
            \end{aligned}
        \right.
        \label{eq:rR_ll_1}
    \end{equation}
    In the same range, on the other hand, $r \gg 1$ and therefore the terms in $1/r^3$ can be omitted Eq. \ref{eq:r_ll_1/R}; the remaining terms are the same as the first terms in Eq. \ref{eq:rR_ll_1}, and the second terms there will be made use of later.

    On going to the next approximation in the near region, write $\mathbf{v} = \mathbf{v}^{(1)} + \mathbf{v}^{(2)}$ and obtain from an equation for the correction in the second approximation:
    \begin{equation}
        \Delta (\nabla \times \mathbf{v}^{(2)}) = - \mathrm{R} \nabla \times (\mathbf{v}^{(1)} \times (\nabla \times \mathbf{v}^{(1)}))
    \end{equation}
    The solution of this equation must satisfy the condition of vanishing on the surface of the sphere and that of joining to the solution in the far region; the latter means that the leading terms in the function $\mathbf{v}^{(2)}(\mathbf{r})$ when $r \gg 1$ must agree with the second term in Eq. \ref{eq:rR_ll_1}. The appropriate solution is
    \begin{align}
        v_r^{(2)} =& \dfrac{3\mathrm{R}}{8} v_r^{(1)} + \dfrac{3\mathrm{R}}{32} \left( 1 - \dfrac1r \right)^2 \left( 2 + \dfrac1r + \dfrac1{r^2} \right) (1 - 3\cos^2 \theta) \nonumber \\
        v_\theta^{(2)} =& \dfrac{3\mathrm{R}}{8} v_\theta^{(1)} + \dfrac{3\mathrm{R}}{32} \left( 1 - \dfrac1r \right)^2 \left( 4 + \dfrac1r + \dfrac1{r^2} + \dfrac2{r^3} \right) \sin\theta\cos\theta \nonumber \\
        r \ll & \dfrac1{\mathrm{R}}
        \label{eq:r_ll_1/R}
    \end{align}
    In the intermediate region, only the terms without a factor $\dfrac1r$ remain in these expressions, and they do in fact with the second terms in Eq.~\ref{eq:rR_ll_1}.

    From the velocity distribution in close region, we can calculate the correction to Stokes formula for the drag. The second terms in Eq.~\ref{eq:r_ll_1/R}, because of their angular dependence, do not contribute to the drag; the first terms give the correction $\dfrac{3\mathrm{R}}8$. According to the above discussion, the exact velocity distribution near the sphere leads in this approximation to the same result for the drag as the solution of Oseen's equation.

    \section{The laminar wake}
    In steady flow of a viscous fluid past a solid body, the flow at great distances behind the body has certain characteristics which can be investigated independently of the particular shape of the body.

    Let us denote by $\mathbf{U}$ the constant velocity of the incident current; we take the direction of $\mathbf{U}$ as the $x$-axis, with the origin somewhere inside the body. The actual fluid velocity at any point may be written $\mathbf{U} + \mathbf{v}$; $\mathbf{v}$ vanishes at infinity.

    It is found that, at great distances behind the body, the velocity $\mathbf{v}$ is noticeably different from zero only in a relatively narrow region near the $x$-axis. This region, called the \emph{laminar wake}, is reached by fluid particles which move along streamlines passing fairly close to the body. Hence the flow in the wake is essentially rotational. The reason is that rotational flow of a viscous fluid past a solid body is due to the surface of the body. This is easily seen if we recall that, in the pattern of potential flow for an ideal flow, only the normal velocity component is zero on the surface of the body, not the tangential component $\mathbf{v}_t$. The boundary condition of adhesion for a real fluid makes $\mathbf{v}_t$ also zero, however. If the pattern of potential flow were maintained, this would cause a non-zero discontinuity of $\mathbf{v}_t$, i.e.\ the occurrence of a surface vorticity. The viscosity smooths out the discontinuity, and the rotational state penetrates into the fluid, from which it passes by convection into the wake region.

    On the other hand, the viscosity has almost no effect at any point on streamlines that do not pass near the body, and the vorticity, which is zero in the incident current, remains parctically zero on these streamlines, as it would in an ideal fluid. Thus the flow at great distances from the body may be regarded as potential flow everywhere except in the wake.

    We shall now derive formulae relating the properties of the flow in the wake to the forces acting on the body. The total momentum transported by the fluid through any closed surface surrounding the body is equal to the integral of the momentum flux density tensor over that surface, $\oint \Pi_{ik} \mathrm{d} f_k$. The components of the tensor $\Pi_{ik}$ are
    \begin{equation*}
        \Pi_{ik} = p \delta_{ik} + \rho (U_i + v_i)(U_k + v_k)
    \end{equation*}
    Write the pressure in the form $p = p_0 + p^\prime$, where $p_0$ is the pressure at infinity. The integration of the constant term $p_0\delta_{ik} + \rho U_i U_k$ gives zeros, since the vector integral $\oint f \mathrm{d} \mathbf{f}$ over a closed surface is zero. The integral $\oint \rho v_k \mathrm{d} f_k$ also vanishes: since the total mass of fluid in the volume considered is zero. Finally, the velocity $\mathbf{v}$ far from the body is small compared with $\mathbf{U}$. Hence, if the surface in question is sufficiently far from the body, we can neglect the term $\rho v_i v_k$ in $\Pi_{ik}$ as compared with $\rho U_k v_i$. Thus the total momentum flux is
    \begin{equation*}
        \oint (p^\prime \delta_{ik} + \rho U_k v_i) \mathrm{d} f_k
    \end{equation*}
    
    Now take the fluid volume concerned to be the volume between two infinite planes $x = \text{const}$, one of them far infront of the body and the other far behind it. The integral over the infinitely distant \emph{lateral} surface vanishes (since $p^\prime = \mathbf{v} = 0$ at infinity), and it is therefore sufficient to integrate only over the two planes. The momentum flux thus obtained is evidently the difference between the total momentum flux entering through the forward plane and that leaving through the backward plane. This difference, however, is just the quantity of momentum transmitted to the body by the fluid per unit time, i.e. the force $\mathbf{F}$ exerted on the body.

    Thus the components of the force $\mathbf{F}$ are
    \begin{align*}
        F_x =& \left( \iint\limits_{x = x_2} - \iint\limits_{x = x_1} \right) (p^\prime + \rho U v_x) \mathrm{d} y \mathrm{d} z \\
        F_y =& \left( \iint\limits_{x = x_2} - \iint\limits_{x = x_1} \right) \rho U v_y \mathrm{d} y \mathrm{d} z \\
        F_z =& \left( \iint\limits_{x = x_2} - \iint\limits_{x = x_1} \right) \rho U v_z \mathrm{d} y \mathrm{d} z
    \end{align*}
    where the integration is taken over the infinite planes $x = x_1$ (far behind the body) and $x = x_2$ (far in front of it). Let us first consider the expression for $F_x$.

    Outside the wake we have potential flow, and therefore Bernoulli's equation
    \begin{equation*}
        p + \dfrac12 \rho (\mathbf{U} + \mathbf{v})^2 = \text{const} \equiv p_0 + \dfrac12 \rho U^2
    \end{equation*}
    holds, or, neglecting the term $\dfrac12 \rho v^2$ in comparison with $\rho \mathbf{U} \cdot \mathbf{v}$,
    \begin{equation*}
        p^\prime = -\rho U v_x
    \end{equation*}
    In this approximation the integrand in $F_x$ vanishes everywhere outside the wake. In other word, the integral over the plane $x = x_2$ (which lies in frnot of the body and does not intersect the wake) is zero, and the integral over the plane $x = x_1$ need be taken only over the area covered by the cross-section of the wake. In the wake, however, the pressure change $p^\prime$ is of the order of $\rho v^2$, i.e. small compared with $\rho Uv_x$. Thus we reach the result that the drag on the body is
    \begin{equation}
        F_x = -\rho U \iint v_x \mathrm{d} y \mathrm{d} z
        \label{eq:Fx}
    \end{equation}
    where the integration is taken over the cross-sectional area of the wake for behind the body. The velocity $v_x$ in the wake is, of course, negative: the fluid moves more slowly than it would if the body were absent. Attention to the fact that the integral in Eq. \ref{eq:Fx} gives the amount by which the discharge through the wake falls short of its value in the absence of the body.
    

    Let us now consider the force (whose components are $F_y$, $F_z$) which tends to move the body transversely. This force is called the \emph{lift}. Outside the wake, where we have potential flow, write $v_y = \partial \phi / \partial y$, $v_z = \partial \phi / \partial z$; the integral over the plane $x = x_2$, which does not meet the wake, is zero:
    \begin{equation*}
        \iint v_y \mathrm{d} y \mathrm{d} z = \iint \dfrac{\partial \phi}{\partial y} \mathrm{d} y \mathrm{d} z = 0, \iint \dfrac{\partial \phi}{\partial z} \mathrm{d} y \mathrm{d} z = 0
    \end{equation*}
    since $\phi = 0$ at infinity. Therefore we find for the lift
    \begin{equation}
        F_y = -\rho U \iint x_y \mathrm{d} y \mathrm{d} z, F_z = -\rho U \iint v_z \mathrm{d} y \mathrm{d} z
    \end{equation}
    The integration in these formulae is again taken only over the cross-sectional area of the wake. If the body has an axis of symmetry (not necessarily complete axial symmetry), and the flow is parallel to this axis, then the flow past the body has an axis of symmetry also. In this case the lift is, of course, zero.

    Let us return to the flow in the wake. An estimate of the magnitudes of various terms in the Navier-Stokes equation shows that the term $\nu \Delta \mathbf{v}$ can in general be neglected at distances $r$ from the body such that $rU/\nu \gg 1$; these are the distances at which the flow outside the wake may be regarded as potential flow. It is not possible to neglect that term inside the wake even at these distances, however, since the transverse derivates $\partial^2 \mathbf{v} / \partial y^2$, $\partial^2 \mathbf{v} / \partial z^2$ are large compared with $\partial^2 \mathbf{v} / \partial x^2$.

    Let $Y$ be of the order of magnitude of the width of the wake, i.e. the distances from the $x$-axis at which the velocity $\mathbf{v}$ falls off markedly. The order of the terms in the Navier-Stokes equation is then
    \begin{equation*}
        (\mathbf{v} \cdot \nabla) \mathbf{v} \sim U \dfrac{\partial U}{\partial x} \sim U \dfrac{v}{x}, \nu \Delta \mathbf{v} \sim \nu \dfrac{\partial^2 v}{\partial y^2} \sim \nu \dfrac{v}{Y^2}
    \end{equation*}
    If these two magnitudes are comparable, we find
    \begin{equation}
        Y = \sqrt{\dfrac{\nu x}{U}}
        \label{eq:comparable_magnitude}
    \end{equation}
    This quantity is in fact small compared with $x$, by the assumed condition $\dfrac{Ux}{\nu} \gg 1$. Thus the width of the laminar wake increases as the square root of the distance from the body.

    In order of determine how the velocity decreases with increasing $x$ in the wake, return to formula Eq. \ref{eq:Fx}. The region of integration has an area of the order of $Y^2$. Hence the integral can be estimated as $F_x \sim \rho U v Y^2$, and by using the relation Eq. \ref{eq:comparable_magnitude} we obtain
    \begin{equation}
        v \sim \dfrac{F_x}{\rho \nu x}
    \end{equation}

    \subsection{Flow inside the wake}
    In the Navier-Stokes equation for steady flow
    \begin{equation}
        (\mathbf{v} \cdot \nabla) \mathbf{v} = - \nabla \left(\dfrac{p}{\rho}\right) + \nu \Delta \mathbf{v}
    \end{equation}
    we use far from the body Oseen's approximation, replacing the term $(\mathbf{v} \cdot \nabla) \mathbf{v}$ by $(\mathbf{U} \cdot \nabla) \mathbf{v}$. Furthermore, inside the wake the derivative with respect to the longitudinal coordinate $x$ in $\Delta \mathbf{v}$ can be neglected in comparison with the transverse derivates. We thus start from the equation
    \begin{equation}
        U \dfrac{\partial \mathbf{v}}{\partial x} = - \nabla \left(\dfrac{p}{\rho}\right) + \nu \left( \dfrac{\partial^2 \mathbf{v}}{\partial y^2} + \dfrac{\partial^2 \mathbf{v}}{\partial z^2}\right)
    \end{equation}

    We seek the solution of this in the form $\mathbf{v} = \mathbf{v}_1 + \mathbf{v}_2$, where $\mathbf{v}_1$ is the solution of
    \begin{equation}
        U \dfrac{\partial \mathbf{v}_1}{\partial x} = \nu \left( \dfrac{\partial \mathbf{v}_1}{\partial y} + \dfrac{\partial^2 \mathbf{v}_1}{\partial z^2} \right)
    \end{equation}
    The quantity $\mathbf{v}_2$ arising from the term $-\nabla \left( \dfrac{p}{\rho} \right)$ in the initial equation may be sought as the gradient of a scalar $\Phi$. Since, far from the body, the derivatives with respect to $x$ are small in comparison with those with respect to $y$ and $z$, in the approximation considered we may neglect the term $\dfrac{\partial \Phi}{\partial x}$, i.e. take $v_x = v_{1x}$. We thus have for $v_x$ the equation
    \begin{equation}
        U \dfrac{\partial v_x}{\partial x} = \nu \left( \dfrac{\partial^2 v_x}{\partial y^2} + \dfrac{\partial^2 v_x}{\partial z^2} \right)
    \end{equation}

    This is formally the same as the two-dimensional equation of the heat conduction, with $x / U$ in place of the time, and the viscosity $\nu$ in place of the thermometric conductivity. The solution which decreases with increasing $y$ and $z$ (for fixed $x$) and gives an infinitely narrow wake as $x \to 0$ (in this approximation the dimensions of the body are regarded as small) is
    \begin{equation}
        v_x = -\dfrac{F_x}{4 \pi \rho \nu x} e^{-\dfrac{U(y^2+z^2)}{4\nu x}}
    \end{equation}
    The constant coefficient in this formula is expressed in terms of the drag , in which the integration may be extended over the whole $yz$-plane because of the rapid convergence. If the Cartesian coordinates are replaced by spherical polar coordinates $r$, $\theta$, $\phi$ with the polar axis along the $x$-axis, then the region of the wake, $\sqrt{(y^2 + z^2)} \ll x$, corresponds to $\theta \ll 1$. In these coordinates, the above formula becomes
    \begin{equation}
        v_x = -\dfrac{F_x}{4 \pi \rho \nu r} e^{- \dfrac{Ur\theta^2}{4\nu}}
    \end{equation}
    The term in $\dfrac{\partial \Phi}{\partial x}$ (with $\Phi$ given by Eq. \ref{eq:Phi} below), which we have omitted, would give a term in $v_x$ which contains an additional small factor $\theta$.

    The form of $v_{1y}$ and $v_{1z}$ must be the same as $v_x$ but with different coefficients. We take the direction of the lift as the $y$axis (so that $F_z = 0$). Since $\Phi = 0$ at infinity,
    \begin{align*}
        \iint v_y \mathrm{d} y \mathrm{d} z =& \iint (v_{1y} + \dfrac{\partial \phi}{\partial y}) \mathrm{d} y \mathrm{d} z \\
        =& \iint v_{1y} \mathrm{d} y \mathrm{d} z \\
        =& - \dfrac{F_y}{\rho U} \\
        \iint v_{1z} \mathrm{d} y \mathrm{d} z =& 0
    \end{align*}
    It is therefore clear that $v_{1y}$ differs from $v_{x}$ in that $F_x$ is replaced by $F_y$, and $v_{1x} = 0$. Thus we find
    \begin{equation}
        v_y = -\dfrac{F_x}{4 \pi \rho \nu x} e^{-\dfrac{U(y^2 + z^2)}{4 \nu x}} + \dfrac{\partial \Phi}{\partial y}, v_z = \dfrac{\partial \Phi}{\partial z}
    \end{equation}

    Next to determine the function $\Phi$. Write the equation of continuity, neglecting the longitudinal derivative:
    \begin{equation}
        \nabla \cdot \mathbf{v} \approx \dfrac{\partial v_y}{\partial y} + \dfrac{\partial v_z}{\partial z} = \left( \dfrac{\partial^2}{\partial y^2} + \dfrac{\partial^2}{\partial z^2} \right) \Phi + \dfrac{\partial v_{1y}}{\partial y} = 0
    \end{equation}
    Differentiating this equation with respect to $x$ for $v_{1y}$,
    \begin{align*}
        \left( \dfrac{\partial^2}{\partial y^2} + \dfrac{\partial^2}{\partial z^2} \right) \dfrac{\partial \Phi}{\partial x} =& -\dfrac{\partial}{\partial y} \left( \dfrac{\partial v_{1y}}{\partial x} \right) \\
        =& -\dfrac{\nu}{U} \left( \dfrac{\partial^2}{\partial y^2} + \dfrac{\partial^2}{\partial z^2} \right) \dfrac{\partial v_{1z}}{\partial y}
    \end{align*}
    Hence
    \begin{equation*}
        \dfrac{\partial \Phi}{\partial x} = - \dfrac{\nu}{U} \dfrac{\partial v_{1y}}{\partial y}
    \end{equation*}
    Finally, substituting the expression for $v_{1y}$, and integrating with respect to $x$,
    \begin{equation}
        \Phi = -\dfrac{F_y}{2 \pi \rho U} \dfrac{y}{y^2 + z^2} \left[ e^{-\dfrac{U(y^2 + z^2)}{4\nu x}} - 1 \right]
        \label{eq:Phi}
    \end{equation}
    the constant of integration is chosen so that $\Phi$ remains finite when $y = z = 0$. In spherical polar coordinates (with the azimuthal angle $\phi$ measured from the $xy$-plane)
    \begin{equation}
        \Phi = -\dfrac{F_y}{2 \pi \rho U} \dfrac{\cos \phi}{r \theta} \left[ e^{-\dfrac{Ur\theta^2}{4\nu}} - 1 \right]
        \label{eq:Phi_theta}
    \end{equation}
    It is seen that $v_y$ and $v_z$, unline $v_x$, contain terms which decrease only as $\dfrac1{\theta^2}$ when we move away from the axis of the wake, as well as those which decrease exponentially with increasing $\theta$ (for a given $r$).

    If there is no lift, the flow in the wake is axially symmetrical, and $\Phi = 0$.

    \subsection{Flow outside the wake}
    Outside the wake, potential flow may be assumed. Since we are interested only in the terms in the potential $\Phi$ which decrease least rapidly at large distances, here seek a solution of Laplace's equation
    \begin{align*}
        \Delta \Phi =& \dfrac1{r^2} \dfrac{\partial}{\partial r} \left(r^2 \dfrac{\partial \Phi}{\partial r} \right) + \dfrac1{r^2 \sin \theta} \dfrac{\partial}{\partial \theta} \left( \sin \theta \dfrac{\partial \Phi}{\partial \theta} \right) \\
        &+ \dfrac1{r^2 \sin^2 \theta}\dfrac{\partial^2 \Phi}{\partial \phi^2} = 0
    \end{align*}
    as a sum of two terms:
    \begin{equation}
        \Phi = \dfrac{a}{r} + \dfrac{\cos \theta}{r} f(\theta)
    \end{equation}
    of which the first is spherically symmetrical and belongs to the force $F_x$, while the second is symmetrical about the $xy$-plane belongs to the force $F_y$.

    For the function $f(\theta)$ the equation
    \begin{equation*}
        \dfrac{\mathrm{d}}{\mathrm{d} \theta} \left( \sin \theta \dfrac{\mathrm{d} f}{\mathrm{d} \theta} \right) - \dfrac{f}{\sin \theta} = 0
    \end{equation*}
    The solution of this equation finite as $\theta \to \pi$ isotropic
    \begin{equation*}
        f = b \cot \dfrac{\theta}2
    \end{equation*}
    The coefficient $b$ must be determined from the condition for joining the solution to that inside the wake. The reason is that Eq. \ref{eq:Phi} relates to the angle range $\theta \ll 1$ and Eq. \ref{eq:Phi_theta} to $\theta \gg \sqrt{\dfrac{\nu}{Ur}}$. These ranges overlap when $\sqrt{\dfrac{\nu}{Ur}} \ll \theta \ll 1$, and Eq. \ref{eq:Phi} then becomes
    \begin{equation*}
        \Phi = \dfrac{F_y}{2\pi \rho U} \dfrac{\cos \phi}{r \theta}
    \end{equation*}
    and the second term in Eq. \ref{eq:Phi_theta} is $\dfrac{2b}{r\theta} \cos \phi$. Comparison of these expressions shows that we must take $b = \dfrac{F_y}{4\pi \rho U}$.

    To determine the coefficient $a$ in \ref{eq:Phi_theta}, since the total mass flux through a sphere $S$ with wlarge radius $r$ equals zero, as for any closed surface. The rate of inflow through the part $S_0$ of $S$ intercepted by the wake is
    \begin{equation*}
        -\iint\limits_{S_0} v_x \mathrm{d} y \mathrm{d} z = \dfrac{F_x}{\rho U}
    \end{equation*}
    Hence the same quantity must flow out through the rest of the surface of the sphere, i.e.
    \begin{equation*}
        \oint\limits_{S - S_0} \mathbf{v} \cdot \mathrm{d} \mathbf{f} = \dfrac{F_x}{\rho U}
    \end{equation*}S
    ince $S-0$ is small compared with $S$,
    \begin{equation*}
        \oint\limits_{S} \mathbf{v} \cdot \mathrm{d} \mathbf{f} = \int\limits_S \nabla \Phi \cdot \mathrm{d} \mathbf{f} = -4\pi a = \dfrac{F_x}{\rho U}
    \end{equation*}
    whence $a = -\dfrac{F_x}{4 \pi \rho U}$.

    The complete expression for the velocity potential is thus
    \begin{equation*}
        \Phi = \dfrac1{4\pi \rho U r}(-F_x + F_y \cos \phi \cot \dfrac12 U)
    \end{equation*}
    which gives the flow everywhere outside the wake far from the body. The potential diminishes with increasing distance $\dfrac1r$, the velocity accordingly decreases as $\dfrac1{r^2}$. If there is no lift, the flow outside the wake is axially symmetrical.

    \section{The viscosity of suspensions}
    A fluid in which numerous fine solid particles are suspended (forming a \emph{suspension}) may be regarded as a homogeneous medium if we are concerned with phenomena whose characteristic lengths are large compared with the dimensions of the particles. Such a medium has an effective viscosity $\eta$ which is different from the viscosity $\eta_0$ of the original fluid. The value of $\eta$ can be calculated for the case where the concentration of the suspended particles is small (i.e. their total volume is small in comparison with that of the fluid). The calculations are relatively simple for the case of spherical particles (A. Einstein).

    It is necessary to consider first the effect of a single solid globule, immersed in a fluid, on flow having a constant velocity gradient. Let the unperturbed flow be described by a linear velocity distribution
    \begin{equation}
        v_{0i} = \alpha_{ik} x_k
        \label{eq:linear_velocity}
    \end{equation}
    where $\alpha_{ik}$ is a constant symmetrical tensor. The fluid pressure is constant:
    \begin{equation*}
        p_0 = \text{const}.
    \end{equation*}
    and take $p_0$ to zero in the following section, i.e. measure only the deviation from this constant value. If the fluid is incompressible ($\nabla \cdot \mathbf{v}_0 = 0$), the sum of the diagonal elements, or trace, of the tensor $\alpha_{ik}$ must be zero:
    \begin{equation}
        \alpha_{ii} = 0
    \end{equation}

    Now let a small sphere with radius $R$ be placed at the origin. Denote the altered fluid velocity by $\mathbf{v} = \mathbf{v}_0 + \mathbf{v}_1$; $\mathbf{v}_1$ must vanish at infinity, but near the sphere $\mathbf{v}_1$ is not small compared with $\mathbf{v}_0$. It is clear from the symmetry of the flow that the sphere remains at rest, so that the boundary condition is $\mathbf{v} = 0$ for $r = R$.

    The required solution of the equations of motion may be obtained at once from the solution of Stokes flow, wiht the function $f$ given previously in Stokes flow. In the present case we desire a solution depending on the components of the tensor $\alpha_{ik}$ as parameters. Such a solution is
    \begin{equation*}
        \mathbf{v}_1 = \nabla \times \nabla \times [(\boldsymbol{\alpha} \cdot \nabla) f], p = \eta_0 \alpha_{ik} \dfrac{\partial^2 \Delta f}{\partial x_i\partial x_k}
    \end{equation*}
    where $(\boldsymbol{\alpha \cdot \nabla}) f$ denotes a vector whose components are $\alpha_{ik} \dfrac{\partial f}{\partial x_k}$. Expanding these expressions and determining the constants $a$ and $b$ in the function $f = ar + \dfrac{b}{r}$ so as to satisfy the boundary conditions at the surface of the sphere, we obtain the following formulae for the velocity and pressure:
    \begin{equation}
        v_{1i} = \dfrac{5}{2} \left( \dfrac{R^5}{r^4} - \dfrac{R^3}{r^2} \right) \alpha_{ki} n_i n_k n_l - \dfrac{R^5}{r^4} \alpha_{ik} n_k
        \label{eq:suspension_velocity}
    \end{equation}
    \begin{equation}
        p = -5 \eta_0 \dfrac{R^3}{r^3} \alpha_{ik} n_i n_k
    \end{equation}
    where $\mathbf{n}$ is a unit vector in the direction of the position vector.

    Returning now to the problem of determining the effective viscosity of a suspension, we calculate the mean value (over the volume) of the momentum flux density tensor $\Pi_{ik}$, which, in the linear approximation with respect to the velocity, is the same as the stress tensor $-\sigma_{ik}$:
    \begin{equation*}
        \delta_{ik} = \dfrac1V \int \sigma_{ik} \mathrm{d} V
    \end{equation*}
    The integration here may be taken over the volume $V$ of a sphere with large radius, which is then extended to infinity.

    First of all,
    \begin{align}
        \delta_{ik} =& \eta_0 \left( \dfrac{\overline{\partial v_i}}{\partial x_k} + \dfrac{\overline{\partial v_k}}{\partial x_i} \right) - \overline{p} \delta_{ik} \nonumber \\
        &+ \dfrac1V \int \left\{ \dfrac{\partial v_i}{\partial x_k} - \eta_0 \left( \dfrac{\partial v_i}{\partial x_k} + \dfrac{\partial v_k}{\partial x_i} \right) + p \delta_{ik} \right\} \mathrm{d} V
        \label{eq:whole_sigma}
    \end{align}
    The integrand on the right is zero except within the solid spheres; since the concentration of the suspension is supposed small, the integral may be calculated for a single sphere as if the others were absent, and then multiplied by the concentration $n$ of the suspension (the number of spheres per unit volume). The direct calculation of this integral would require an investigation of internal stresses in the spheres. We can circumvent this difficulty, however, by transforming the volume integral into a surface integral over an infinitely distant sphere, which lies entirely in the fluid. To do so, the equation of motion $\dfrac{\partial \sigma_{il}}{\partial x_l} = 0$ leads to the identity
    \begin{equation*}
        \sigma_{ik} = \dfrac{\partial (\sigma_{il} x_k)}{\partial x_l}
    \end{equation*}
    hence the transformation of the volume integral into a surface integral gives
    \begin{equation*}
        \overline{\sigma}_{ik} = \eta_0 \left( \dfrac{\overline{\partial v_i}}{\partial x_k} + \dfrac{\overline{\partial v_k}}{\partial x_i} \right) + n \oint [\sigma_{il} x_k \mathrm{d} f_l - \eta_0 (v_i \mathrm{d} f_k + v_k \mathrm{d} f_i)]
    \end{equation*}
    Omit the term in $\overline{p}$, since the mean pressure is necessarily zero; $\overline{p}$ is a scalar which must be given by a linear combination of the components $\alpha_{ik}$, and the only such scalar is $\alpha_{ii} = 0$.

    When calculating the integral over a sphere with very large radius, only the terms of order $1/r^2$ need be retained in Eq. \ref{eq:suspension_velocity}. A simple calculation gives the value of the integral as
    \begin{equation*}
        n \eta_0 \cdot 20 \pi R^3 (5 \alpha_{lm} \overline{n_i n_k n_l n_m} - \alpha_{il} \overline{n_k n_l})
    \end{equation*}
    where the bar denotes an average with respect to directions of the unit vector $\mathbf{n}$. Effecting the averaging, finally
    \begin{equation}
        \overline{\sigma}_{ik} = \eta_0 \left( \dfrac{\overline{\partial v_i}}{\partial x_k} + \dfrac{\overline{\partial v_k}}{\partial x_i} \right) + 5 \eta_0 \alpha_{ik} \cdot \dfrac43 \pi R^3 n
        \label{eq:averaged_sigma}
    \end{equation}

    The first term in Eq. \ref{eq:averaged_sigma}, on substitution of $v_0$ from Eq. \ref{eq:linear_velocity}, gives $2 \eta_0 \alpha_{ik}$; the first-order small component is identically zero after averaging with respect to the directions of $\mathbf{n}$, as it should be, since the effect resides entirely in the integral separated in Eq. \ref{eq:whole_sigma}. Hence the required relative correction to the effective viscosity $\eta$ of the suspension is determined by the ratio-of the second and the first terms in Eq. \ref{eq:averaged_sigma}. Thus
    \begin{equation}
        \eta = \eta_0 (1 + \dfrac52 \phi), \phi = \dfrac43 \pi R^3 n
    \end{equation}
    where $\phi$ is the small ratio of the total volume of the sphere to the total volume of the suspension.

    In the flow of a suspension of non-spherical particles, the presence of velocity gradients has an orienting effect on them. The simultaneous action of orienting hydrodynamics forces and disorienting rotary Brownian motion gives rise to an anisotropic distribution of the particles as regards their orientation in space. This, however, need not be considered when calculating the correction to the viscosity $\eta$: the anisotropy of the orientation distribution is itself dependent on the velocity gradients (linearly in the first approximation), and including it would give stress tensor terms non-linear in the gradients.

    \section{Oscillatory motion in a viscous fluid}
    When a solid body immersed in a viscous fluid oscillates, the flow thereby set up has a number of characteristic properties. A simple but typical example (G. G. Stokes 1851) is presented first. Suppose an incompressible fluid is bounded by an infinite plane surface which executes a simple harmonic oscillation in its own plane, with frequency $\omega$. Take the solid surface as the $yz$-plane, and the fluid region as $x > 0$; the 
\end{document}